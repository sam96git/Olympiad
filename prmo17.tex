\documentclass{article}
\usepackage{gvv}
\date{}
\title{Olympiad}

\begin{document}
\maketitle
\begin{enumerate}

\item How many positive integers less than $1000$ have the property that the sum of the digits of each such number is divisible by $7$ and the number itself is divisible by $3$?
\item Suppose $a,b$ are positive real numbers such that $a\sqrt{a}+b\sqrt{b}=183,\ a\sqrt{b}+b\sqrt{a}=182$. Find $\frac{9}{5}\brak{a+b}$.
\item A contractor has two teams of workers: team $A$ and team $B$. Team $A$ can complete a job in $12$ days and team $B$ can do the same job in $36$ days. Team $A$ starts working on the job and team $B$ joins team $A$ after four days. The team $A$ withdraws after two more days. For how many more days should team $B$ work to complete the job?
\item Let $a,b$ be integers such that all the roots of the equation $\brak{x^{2}+ax+20}\brak{x^{2}+17x+b}=0$ are negative integers. What is the smallest possible value of $a+b$ ?
\item Let $u,v,w$ be real numbers in geometric progression such that $u>v>w$. Suppose $u^{40}=v^{n}=w^{60}$. Find the value of $n$.
\item Let the sum $\sum_{n=1}^{9}\frac{1}{n\brak{n+1}\brak{n+2}}$ written in its lowest terms be $\frac{p}{q}$. Find the value of $q-p$.
\item Find the number of positive integers $n$, such that $\sqrt{n}+\sqrt{n+1}<11$.
\item A pen costs $\$ 11$ and a notebook costs $\$ 13$. Find the number of ways in which a person can spend exactly $\$ 1000$ to buy pens and notebooks.
\item There are five cities $A, B, C, D, E$ on a certain island. Each city is connected to every other city by road. In how many ways can a person starting from city $A$ come back to $A$ after visiting some cities without visiting a city more than once and without taking the same road more than once? (The order in which he visits the cities also matters: e.g., the routes $A\rightarrow B\rightarrow C\rightarrow A$ and A $\rightarrow C \rightarrow B \rightarrow A$ are different.)
\item There are eight rooms on the first floor of a hotel, with four rooms on each side of the corridor, symmetrically situated (that is each room is exactly opposite to one other room). Four guests have to be accommodated in four of the eight rooms (that is, one in each) such that no two guests are in adjacent rooms or in opposite rooms. In how many ways can the guests be accommodated?
\item Let $f\brak{x}=\sin\frac{x}{3}+\cos\frac{3x}{10}$ for all real $x$. Find the least natural number $n$ such that $f\brak{n\pi+x}=f\brak{x}$ for all real $x$.
\item In a class, the total numbers of boys and girls are in the ratio $4:3$. On one day it was found that $8$ boys and $14$ girls were absent from the class, and that the number of boys was the square of the number of girls. What is the total number of students in the class?
\item In a rectangle $ABCD$, $E$ is the midpoint of $AB$; $F$ is a point on $AC$ such that $BF$ is perpendicular to $AC$; and $FE$ perpendicular to $BD$. Suppose $BC = 8\sqrt{3}$. Find $AB$.
\item Suppose $x$ is a positive real number such that $\cbrak{x}, \sbrak{x}$ and $x$ are in a geometric progression. Find the least positive integer $n$ such that $x^{n}>100$. (Here $\sbrak{x}$ denotes the integer part of $x$ and $\cbrak{x}=x-\sbrak{x}$.) 
\item Integers $1, 2, 3,\ldots,n,$ where $n>2$, are written on a board. Two numbers $m,k$ such that $1<m<n,\ 1<k<n$ are removed and the average of the remaining numbers is found to be $17$. What is the maximum sum of the two removed numbers?
\item Five distinct 2-digit numbers are in a geometric progression. Find the middle term.
\item Suppose the altitudes of a triangle are $10, 12$ and $15$. What is its semi-perimeter?
\item If the real numbers $x,y,z$ are such that $x^2+4y^2+16z^2 = 48$ and $xy+4yz+2zx = 24$, what is the value of $x^2+y^2+z^2$?
\item Suppose $1,2,3$ are the roots of the equation $x^4+ax^2+bx=c$. Find the value of $c$.
\item What is the number of triples $\brak{a, b, c}$ of positive integers such that $(i)\ a<b<c<10$ and $(ii)\ a,b,c,10$ form the sides of a quadrilateral?
\item Find the number of ordered triples $\brak{a, b, c}$ of positive integers such that $abc = 108$.
\item Suppose in the plane $10$ pairwise nonparallel lines intersect one another. What is the maximum possible number of polygons (with finite areas) that can be formed?
\item Suppose an integer $x$, a natural number $n$ and a prime number $p$ satisfy the equation $7x^2-44x+12=p^n$. Find the largest value of the $p$.
\item Let $P$ be an interior point of a triangle $ABC$ whose sidelengths are $26, 65, 78$. The line through $P$ parallel to $BC$ meets $AB$ in $K$ and $AC$ in $L$. The line through $P$ parallel to $CA$ meets $BC$ in $M$ and $BA$ in $N$. The line through $P$ parallel to $AB$ meets $CA$ in $S$ and $CB$ in $T$. If $KL, MN, ST$ are of equal lengths, find this common length.
\item Let $ABCD$ be a rectangle and let $E$ and $F$ be points on $CD$ and $BC$ respectively such that area$\brak{ADE} = 16,\ area\brak{CEF}=9$ and area$\brak{ABF} = 25$. What is the area of triangle $AEF$?
\item Let $AB$ and $CD$ be two parallel chords in a circle with radius $5$ such that the centre $O$ lies between these chords. Suppose $AB = 6,\ CD = 8$. Suppose further that the area of the part of the circle lying between the chords $AB$ and $CD$ is $(m\pi+n)/k$, where $m, n, k$ are positive integers with $gcd\brak{m, n, k}=1$. What is the value of $m + n + k$?
\item Let $\Omega_1$ be a circle with centre $O$ and let $AB$ be a diameter of $\Omega_1$. Let $P$ be a point on the segment $OB$ different from $O$. Suppose another circle $\Omega_2$ with centre $P$ lies in the interior of $\Omega_1$. Tangents are drawn from $A$ and $B$ to the circle $\Omega_2$ intersecting $\Omega_1$ again at $A_1$ and $B_1$ respectively such that $A_1$and $B_1$ are on the opposite sides of $AB$. Given that $A_1B = 5,\ AB_1 = 15$ and $OP = 10$, find the radius of $\Omega_1$.
\item Let $p, q$ be prime numbers such that $n^{3pq}-n$ is a multiple of $3pq$ for all positive integers $n$. Find the least possible value of $p + q$.
\item For each positive integer $n$, consider the highest common factor $h_n$ of the two numbers $n!+1$ and $\brak{n + 1}!$. For $n < 100$, find the largest value of $h_n$.
\item Consider the areas of the four triangles obtained by drawing the diagonals $AC$ and $BD$ of a trapezium $ABCD$. The product of these areas, taken two at time, are computed. If among the six products so obtained, two products are $1296$ and $576$, determine the square root of the maximum possible area of the trapezium to the nearest integer. 

\end{enumerate}
\end{document}
